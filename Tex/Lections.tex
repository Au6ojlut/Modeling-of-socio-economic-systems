\documentclass[12pt,a4paper,titlepage,oneside]{book}
\usepackage[utf8]{inputenc}
\usepackage[russian]{babel}
\usepackage[OT1]{fontenc}
\usepackage{amsmath}
\usepackage{amsthm}
\usepackage{mathrsfs}
\usepackage{indentfirst}
\usepackage{amsfonts}
\usepackage{bbold}
\usepackage[hidelinks]{hyperref}
\usepackage{amssymb}
\usepackage[left=2cm,right=2cm,top=2cm,bottom=2cm]{geometry}
\title{Функциональный анализ}

%%% COMMANDS
\newcommand{\overbar}[1]{\mkern 1.5mu\overline{\mkern-1.5mu#1\mkern-1.5mu}\mkern 1.5mu}
\newcommand{\argmax}{\operatornamewithlimits{argmax}}
\newcommand{\argmin}{\operatornamewithlimits{argmin}}

%%% THEOREMS
\newtheoremstyle{break}{3pt}{3pt}{\itshape}{}{\bfseries}{.}{\newline}{}

\theoremstyle{definition}
\newtheorem{definition}{Определение}[chapter]

\theoremstyle{plain}
\newtheorem{theorem}{Теорема}[chapter]

\theoremstyle{remark}
\newtheorem{remark}{Замечание}[chapter]

\theoremstyle{remark}
\newtheorem{example}{Пример}[chapter]

\theoremstyle{plain}
\newtheorem{lemma}{Лемма}[chapter]

\theoremstyle{plain}
\newtheorem{corollary}{Следствие}[chapter]

%%% MISC
\setcounter{tocdepth}{1}
\def\labelitemi{--}
\renewcommand{\qedsymbol}{\rule{0.7em}{0.7em}}
\sloppy
\binoppenalty=\maxdimen
\relpenalty=\maxdimen

%%% DOCUMENT
\begin{document}

%%% TITLE PAGE
\begin{titlepage}
\begin{center}

\vfill

Санкт-Петербургский государственный университет\\
\ \\

\vfill

{\large\bf Моделирование социально-экономических систем\\}
\ \\
Лекции 
\vfill

\hfill\vbox
{
\hbox{Доцент кафедры математического моделирования}
\hbox{энергетических систем, кандидат физ.-мат. наук}
\hbox{Александр Юрьевич Крылатов}
}

\vfill

Санкт-Петербург, 2016
\end{center}
\end{titlepage}


\tableofcontents

\chapter{Балансовая модель производства}

\section{Модель <<затрата - выпуск>> (англ. input - output)}

Предположим следующее:
\begin{enumerate}

\item[1)]Количество продукции характеризуется одним числом (у каждого экономического объекта).
\item[2)]Комплектность потребления: для выпуска продукции экономический объект должен получить продукты от других объектов.
\item[3)]Линейность : для увеличения количества производства в $n$ раз, необходимо увеличить ресурс в $n$ раз.
\item[4)]Делимость на конечный продукт и на продукт, который будет использоваться в производстве.

\end{enumerate}

$\\ $

Пусть $n$ --- количество субъектов (экономических субъектов),

$x_i$ --- количество производства продукта $i$,

$\\ $

$x$ = $\left[\begin{array}{crl}
x_1\\ x_2\\ ... \\ x_n
\end{array}\right]$,

$\\ $

$x_{ji}$ --- количество продукта j, необходимого для производства i.

$\\ $

$\begin{cases}
x_{1i} = \alpha_{1i} x_i \\
x_{2i} = \alpha_{2i} x_i \\
... \\
x_{ni} = \alpha_{ni} x_i
\end{cases}$

$\\ $

$A=\left[\begin{array}{crl}
\alpha_{11} & ... & \alpha_{1n} \\
... & ... & ...\\
\alpha_{n1} & ... & \alpha_{nn}
\end{array}\right]$

$\\ $

\begin{definition}
$A$ --- матрица коэффициентов прямых затрат (матрица технологических коэффициентов).


Матрица $A$ --- положительно полуопределённая ($z^T A z   \geq 0$, для любых ненулевых векторов $z$).
\end{definition}

$y_i$ -- количество $i$-го продукта на продажу.
$$\sum \limits_{i = 1}^{n} \alpha_{ji}x_i + y_i = x_j \qquad \forall j = \bar{1,n};$$
$$Ax+y=x \leftrightarrow y=(E - A) x$$
$$x=(E-A)^{-1}y$$
Для того, чтобы это уравнение имело единственное решение необходимо и достаточно, чтобы  $det(E-A) \neq 0$.
$x_j \geq 0 \qquad \forall j$
\begin{remark}
Далее под обозначением $x \geq 0$ будем понимать покомпонентную неотрицательность вектора $x$.
\end{remark}
\begin{definition}
Квадратная матрица $A$, такая, что $A_{ij} \geq 0 \qquad \forall i,j$, называется продуктивной, если существует хотя бы один такой вектор $\bar{x} > 0$, что $(E-A)\bar{x} > 0$.
\end{definition}
\begin{theorem}(О существовании и единственности решение балансовой системы уравнений)\label{t1.1}
Матрица $A$ продуктивна, тогда и только тогда, когда существует, единственно и неотрицательно решение системы $(E-A)x=y$ для любого  вектора $y \geq 0$.
\end{theorem}
\begin{proof}
\textit{Достаточность.}

Рассмотрим $\bar{y} > 0$ и $\bar{x} \geq 0$. $(E-A)\bar{x} = \bar{y} > 0 \quad \rightarrow \quad \bar{x} > A\bar{x} \quad \rightarrow \quad \bar{x} \geq 0$. $(E-A)\bar{x} > 0$ 
\begin{lemma}\label{l1.1}
Если $A$ -- продуктивна, то
$$\lim_{\nu \to \infty} A^{\nu} = 0 \qquad \nu \in N$$

\end{lemma}
\begin{proof}
$ \bar{x} \overset{\mathrm{def}}{>} A \bar{x} \geq 0$. Существует $\lambda : 0 < \lambda < 1$ такая, что $$\lambda \bar{x} > A\bar{x}.$$ 
Домножим обе части на $A$:
$$\lambda A \bar{x} \geq A^2 \bar{x} \geq 0$$
А теперь на $\lambda$:
$$ \lambda^2 \bar{x} > \lambda A \bar{x} \geq 0$$
Не трудно увидеть, что $\lambda^2 \bar{x} > A^2 \bar{x} \geq 0$. Тогда продолжая этот процесс получим 
$$\lambda^{\nu} \bar{x} > A^{\nu} \bar{x} \geq 0.$$
Так как $\lambda^{\nu} \to 0$ при $ \nu \to \infty$, то $A^{\nu} \to 0 $ при $ \nu \to \infty.$
\end{proof}

\begin{lemma}\label{l1.2}
Если $A$ -- продуктивна и существует такой вектор $\bar{x}$, что выполняется $\bar{x} \geq A\bar{x}$, то $\bar{x} \geq 0.$
\end{lemma}
\begin{proof}

$$\bar{x} \geq A\bar{x} \geq A^2\bar{x} \geq \dots \geq A^{\nu} \bar{x}$$
$$\bar{x} \geq A^{\nu} \bar{x} \to 0 \text{, при } \nu \to \infty$$
$$\bar{x} \geq 0$$
\end{proof}

\begin{lemma}\label{l1.3}
Если $A$ -- продуктивна, то $det(E-A) \neq 0.$
\end{lemma}
\begin{proof}
\textit{От противного.}\\
Если $A$ -- продуктивна, но $det(E-A) = 0.$\\
Пусть существует такой вектор $\hat{x} \neq 0$, и пусть $(E-A)\hat{x} = 0 \quad \overset{\mathrm{Lemma 1.2}}{\Longrightarrow} \quad \hat{x} \geq 0$.\\
Теперь возьмем вектор $(-\hat{x}),$ $(E-A)(-\hat{x}) = 0 \quad \overset{\mathrm{Lemma 1.2}}{\Longrightarrow} \quad (-\hat{x}) \geq 0.$
Пришли к противоречию.
\end{proof}

\textit{Необходимость.}

$$(E-A)x=y \quad \forall y \geq 0$$
По Лемме 1.3 $det(E-A) \neq 0$, следовательно решение единственно.
$$(E-A)x \geq 0$$
В силу Леммы 1.2 $x \geq 0.$
\end{proof}

\begin{theorem}\label{t1.2}
Матрица $A \geq 0$ -- продуктивна тогда и только тогда, когда $S = (E-A)^{-1}$ существует и не отрицательна.
\end{theorem}

\begin{proof}

\textit{Необходимость.}

$S = \{ \sigma_{ij} \}_{i}^{j}$

Рассмотрим $(E-A)x = u_j$, где

$u_j = \left(\begin{array}{crl}
0\\ ... \\1\\...\\ 0
\end{array}\right)$ (Единица на $j$-ом месте)

В силу теоремы ($\ref{t1.1}$) $(E-A)x=y$ имеет единственное решение $x=Sy$, следовательно $\sigma_{ij} \geq 0 \quad \forall i = \overline{1,n}$.

\textit{Достаточность.}

Рассмотрим $\hat{x}: \quad (E-A)^{-1}u$

$u = \left(\begin{array}{crl}
1\\ ... \\1\\...\\ 1
\end{array}\right)$

$\hat{x} = \sum\limits_{j=1}^{n} \sigma_{ij}$

Так как $|E-A| \neq 0, (E-A)^{-1}$ -- ни один столбец не состоит из нулей. Тогда 

$\hat{x} = \sum\limits_{j=1}^{n} \sigma_{ij} > 0$

$(E-A)\hat{x}=u>0$

$S=(E-A)^{-1}$
\end{proof}

\begin{definition}
Компоненты матрицы $S$ называются коэффициентами полезных затрат, а $S$ -- матрица коэффициентов полных затрат.

$x = Sy$

\end{definition}


Составление плана не ясно, нужны комментарии.

\begin{theorem}
Если $A$ продуктивная, то $\lim\limits_{\nu \to \infty} y_{\nu} = (E-A)^{-1}y_0$.
\end{theorem}

\begin{proof}
$y_{\nu} = (E-A)^{-1}y_0 - (E-A)^{-1}A^{\nu+1}y_0 \to (E-A)^{-1}y_0$
\end{proof}



Пример про составление плана с двумя определениями не понятно. Спросить.
\chapter{Линейное программирование}

[LECTION N4]

[Здесь могла быть ваша реклама]
$F = \sum\limits_{j=1}^m F^j$ -- имеется $m$ групп пользователей,
$F^j$ -- поток группы.\\ 
Каждая группа стремиться минимизировать совой поток:
\begin{gather*}
\min\limits_{f^j} \sum\limits_{i=1}^n t_i(f_i)f_i^j, \qquad f^j = (f_1^j, f_2^j, \dots, f_n^j) - \text{стратегия} \\
\sum\limits_{i=1}^n f_i^j = F^j, \qquad f_i = \sum\limits_{j=1}^m f_i^j
\end{gather*}

\begin{equation*}
\frac{\partial L^j}{\partial f_i^j} = t_i(f_i)+ \frac{\partial t_i(f_i)}{\partial f_i^j}f_i^j - \omega^j-\eta_i^j = 0
\end{equation*}

\begin{equation*}
t_i(f_i) + \frac{\partial t_i(f_i)}{\partial f_i^j}f_i^j \begin{cases} = \omega^j & если f_i^j > 0\\
> \omega^j & если f_i^j = 0 
\end{cases}
\end{equation*}

\begin{equation*}
t_i(f_i) = a_i + b_if_i
\end{equation*}

\begin{equation*}
a_i+b_i \sum\limits_{j=1}^m f_i^j + b_i f_i^j \begin{cases} = \omega^j & если f_i^j > 0\\
\geq \omega^j & если f_i^j = 0 
\end{cases}
\end{equation*}

\begin{equation*}
f_i^1+ \dots + 2f_i^j+ \dots +  f_i^m  \begin{cases} = \frac{\omega^j -a_i}{b_i} & если f_i^j > 0\\
\geq \frac{\omega^j -a_i}{b} & если f_i^j = 0 \text{!!!!!!!!!!!}
\end{cases}
\end{equation*}

\begin{equation*}
\begin{pmatrix}
2 & 1 & \cdots & 1\\
1 & 2 & \cdots & 1\\
\vdots & \vdots & \ddots & \vdots\\
1 & 1 & \cdots & 2 
\end{pmatrix}
\begin{pmatrix}
f_i^1 \\ f_i^2 \\ \vdots\\ f_i^m 
\end{pmatrix}
=
\begin{pmatrix}
\frac{\omega^1 - a_i}{b_i} \\ \frac{\omega^2 - a_i}{b_i} \\ \vdots \\ \frac{\omega^m - a_i}{b_i}
\end{pmatrix}
\end{equation*}
Следовательно
\begin{equation*}
\begin{pmatrix}
f_i^1 \\ f_i^2 \\ \vdots\\ f_i^m 
\end{pmatrix} 
=\begin{pmatrix}
\frac{m}{m+1} & \frac{-1}{m+1} & \cdots & \frac{-1}{m+1} \\
\frac{-1}{m+1} & \frac{m}{m+1} & \cdots & \frac{-1}{m+1} \\
\vdots & \vdots & \ddots & \vdots\\
\frac{-1}{m+1} & \frac{-1}{m+1} & \cdots & \frac{m}{m+1} \\
\end{pmatrix}
\begin{pmatrix}
\frac{\omega^1 - a_i}{b_i} \\ \frac{\omega^2 - a_i}{b_i} \\ \vdots \\ \frac{\omega^m - a_i}{b_i}
\end{pmatrix}
\end{equation*}

Для экономии времени и места обозначим $\xi_i^j = \frac{\omega^j - a_i}{b_i}$.

\begin{equation*}
f_i^j = \xi_i^j - \frac{1}{m+1} \sum\limits_{q=1}^m \xi_i^q
\end{equation*}

\begin{equation}
F^j = \sum\limits f_i^j = \sum\limits_{i=1}^n \xi_i^j - \frac{1}{m+1} \sum\limits_{i=1}^n \sum\limits_{q=1}^m \xi_i^q
\end{equation}

\begin{equation*}
\begin{pmatrix}
F^1 \\ F^2 \\ \vdots\\ F^m 
\end{pmatrix} 
=
\begin{pmatrix}
\frac{m}{m+1} & \frac{-1}{m+1} & \cdots & \frac{-1}{m+1} \\
\frac{-1}{m+1} & \frac{m}{m+1} & \cdots & \frac{-1}{m+1} \\
\vdots & \vdots & \ddots & \vdots\\
\frac{-1}{m+1} & \frac{-1}{m+1} & \cdots & \frac{m}{m+1} \\
\end{pmatrix}
\begin{pmatrix}
\sum\limits_{i=1}^n \xi_i^1 \\ \sum\limits_{i=1}^n \xi_i^2 \\ \vdots \\ \sum\limits_{i=1}^n \xi_i^m
\end{pmatrix}
\end{equation*}

\begin{equation*}
\begin{pmatrix}
\sum\limits_{i=1}^n \xi_i^1 \\ \sum\limits_{i=1}^n \xi_i^2 \\ \vdots \\ \sum\limits_{i=1}^n \xi_i^m
\end{pmatrix}
=
\begin{pmatrix}
2 & 1 & \cdots & 1\\
1 & 2 & \cdots & 1\\
\vdots & \vdots & \ddots & \vdots\\
1 & 1 & \cdots & 2 
\end{pmatrix}
\begin{pmatrix}
F^1 \\ F^2 \\ \vdots\\ F^m 
\end{pmatrix}
\end{equation*}
Отсюда

\begin{equation*}
\sum\limits_{i=1}^n\xi_i^j = F^1 + \dots + 2F^j+ \dots + F^m
\end{equation*}

\begin{equation*}
\sum\limits_{i=1}^n\frac{\omega^j - a_i}{b_i} = F^j + \sum\limits_{i=1}^m F^i
\end{equation*}

\begin{equation*}
\omega^j = \frac{F^j + \sum\limits_{i=1}^m F^i + \sum\limits_{i=1}^n  \frac{a_i}{b_i}}{\sum\limits_{i=1}^n \frac{1}{b_i}}
\end{equation*}

\begin{equation*}
\xi_i^j = \frac{1}{b_i}\frac{F^j + \sum\limits_{q=1}^m F^q + \sum\limits_{s=1}^n \frac{a_s}{b_s}}{\sum\limits_{s=1}^n \frac{1}{b_s}} - \frac{a_i}{b_i}
\end{equation*}

Если $m=1$, то
\begin{equation*}
f_i^j = \frac{1}{b_i}\frac{F+ \frac{1}{2}\sum\limits_{s=1}^n \frac{a_s}{b_s}}{\sum\limits_{s=1}^n\frac{1}{b_s}} + \frac{1}{2}\frac{a_i}{b_i}
\end{equation*}

\begin{equation*}
T^{u \varepsilon}(f_i) = \sum\limits_{i=1}^n \int\limits_0^{f_i} t_i(u)du
\end{equation*}

\begin{equation*}
T^{so}(f_i) = \sum\limits_{i=1}^n t_i(f_i)f_i
\end{equation*}


\begin{equation*}
T_m^{n \varepsilon}(f_i) = \sum\limits_{j=1}^m \sum\limits_{i=1}^n t_i(f_i)f_i^j
\end{equation*}

\begin{equation*}
T^{so} \leq T_m^{n \varepsilon} \leq T^{u \varepsilon}
\end{equation*}

\begin{equation*}
T^{so} = T_1^{n \varepsilon} \leq T_2^{n \varepsilon} \leq \dots \leq T_{|F|}^{n \varepsilon} 
= T_{\infty}^{n \varepsilon} = T^{u \varepsilon} [!!!!!!!!!!!!]
\end{equation*}

\chapter{Нелинейное программирование))))))}


\chapter{Рекомендуемая литература}

\begin{enumerate}
\item <<Новое индустриальное общество>>, Джон Кеннет Гэлбрейт
\end{enumerate}
\end{document}
