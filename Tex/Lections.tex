\documentclass[12pt,a4paper,titlepage,oneside]{book}
\usepackage[utf8]{inputenc}
\usepackage[russian]{babel}
\usepackage[OT1]{fontenc}
\usepackage{amsmath}
\usepackage{amsthm}
\usepackage{mathrsfs}
\usepackage{indentfirst}
\usepackage{amsfonts}
\usepackage{bbold}
\usepackage[hidelinks]{hyperref}
\usepackage{amssymb}
\usepackage[left=2cm,right=2cm,top=2cm,bottom=2cm]{geometry}
\title{Функциональный анализ}

%%% COMMANDS
\newcommand{\overbar}[1]{\mkern 1.5mu\overline{\mkern-1.5mu#1\mkern-1.5mu}\mkern 1.5mu}
\newcommand{\argmax}{\operatornamewithlimits{argmax}}
\newcommand{\argmin}{\operatornamewithlimits{argmin}}

%%% THEOREMS
\newtheoremstyle{break}{3pt}{3pt}{\itshape}{}{\bfseries}{.}{\newline}{}

\theoremstyle{definition}
\newtheorem{definition}{Определение}[chapter]

\theoremstyle{plain}
\newtheorem{theorem}{Теорема}[chapter]

\theoremstyle{remark}
\newtheorem{remark}{Замечание}[chapter]

\theoremstyle{remark}
\newtheorem{example}{Пример}[chapter]

\theoremstyle{plain}
\newtheorem{lemma}{Лемма}[chapter]

\theoremstyle{plain}
\newtheorem{corollary}{Следствие}[chapter]

%%% MISC
\setcounter{tocdepth}{1}
\def\labelitemi{--}
\renewcommand{\qedsymbol}{\rule{0.7em}{0.7em}}
\sloppy
\binoppenalty=\maxdimen
\relpenalty=\maxdimen

%%% DOCUMENT
\begin{document}

%%% TITLE PAGE
\begin{titlepage}
\begin{center}

\vfill

Санкт-Петербургский государственный университет\\
\ \\

\vfill

{\large\bf Моделирование социально-экономических систем\\}
\ \\
Лекции 
\vfill

\hfill\vbox
{
\hbox{Доцент кафедры математического моделирования}
\hbox{энергетических систем, кандидат физ.-мат. наук}
\hbox{Александр Юрьевич Крылатов}
}

\vfill

Санкт-Петербург, 2016
\end{center}
\end{titlepage}


\tableofcontents

\chapter{Балансовая модель производства}

\section{Модель <<затрата - выпуск>> (англ. input - output)}

Предположим следующее:
\begin{enumerate}

\item[1)]Количество продукции характеризуется одним числом (у каждого экономического объекта).
\item[2)]Комплектность потребления: для выпуска продукции экономический объект должен получить продукты от других объектов.
\item[3)]Линейность : для увеличения количества производства в $n$ раз, необходимо увеличить ресурс в $n$ раз.
\item[4)]Делимость на конечный продукт и на продукт, который будет использоваться в производстве.

\end{enumerate}

$\\ $

Пусть $n$ --- количество субъектов (экономических субъектов),

$x_i$ --- количество производства продукта $i$,

$\\ $

$x$ = $\left[\begin{array}{crl}
x_1\\ x_2\\ ... \\ x_n
\end{array}\right]$,

$\\ $

$x_{ji}$ --- количество продукта j, необходимого для производства i.

$\\ $

$\begin{cases}
x_{1i} = \alpha_{1i} x_i \\
x_{2i} = \alpha_{2i} x_i \\
... \\
x_{ni} = \alpha_{ni} x_i
\end{cases}$

$\\ $

$A=\left[\begin{array}{crl}
\alpha_{11} & ... & \alpha_{1n} \\
... & ... & ...\\
\alpha_{n1} & ... & \alpha_{nn}
\end{array}\right]$

$\\ $

\begin{definition}
$A$ --- матрица коэффициентов прямых затрат (матрица технологических коэффициентов).


Матрица $A$ --- положительно полуопределённая ($z^T A z   \geq 0$, для любых ненулевых векторов $z$).
\end{definition}

$y_i$ -- количество $i$-го продукта на продажу.
$$\sum \limits_{i = 1}^{n} \alpha_{ji}x_i + y_i = x_j \qquad \forall j = \bar{1,n};$$
$$Ax+y=x \leftrightarrow y=(E - A) x$$
$$x=(E-A)^{-1}y$$
Для того, чтобы это уравнение имело единственное решение необходимо и достаточно, чтобы  $det(E-A) \neq 0$.
$x_j \geq 0 \qquad \forall j$
\begin{remark}
Далее под обозначением $x \geq 0$ будем понимать покомпонентную неотрицательность вектора $x$.
\end{remark}
\begin{definition}
Квадратная матрица $A$, такая, что $A_{ij} \geq 0 \qquad \forall i,j$, называется продуктивной, если существует хотя бы один такой вектор $\bar{x} > 0$, что $(E-A)\bar{x} > 0$.
\end{definition}
\begin{theorem}(О существовании и единственности решение балансовой системы уравнений)
Матрица $A$ продуктивна, тогда и только тогда, когда существует, единственно и неотрицательно решение системы $(E-A)x=y$ для любого  вектора $y \geq 0$.
\end{theorem}
\begin{proof}
\textit{Достаточность.}

Рассмотрим $\bar{y} > 0$ и $\bar{x} \geq 0$. $(E-A)\bar{x} = \bar{y} > 0 \quad \rightarrow \quad \bar{x} > A\bar{x} \quad \rightarrow \quad \bar{x} \geq 0$. $(E-A)\bar{x} > 0$ 
\begin{lemma}
Если $A$ -- продуктивна, то
$$\lim_{\nu \to \infty} A^{\nu} = 0 \qquad \nu \in N$$

\end{lemma}
\begin{proof}
$ \bar{x} \overset{\mathrm{def}}{>} A \bar{x} \geq 0$. Существует $\lambda : 0 < \lambda < 1$ такая, что $$\lambda \bar{x} > A\bar{x}.$$ 
Домножим обе части на $A$:
$$\lambda A \bar{x} \geq A^2 \bar{x} \geq 0$$
А теперь на $\lambda$:
$$ \lambda^2 \bar{x} > \lambda A \bar{x} \geq 0$$
Не трудно увидеть, что $\lambda^2 \bar{x} > A^2 \bar{x} \geq 0$. Тогда продолжая этот процесс получим 
$$\lambda^{\nu} \bar{x} > A^{\nu} \bar{x} \geq 0.$$
Так как $\lambda^{\nu} \to 0$ при $ \nu \to \infty$, то $A^{\nu} \to 0 $ при $ \nu \to \infty.$
\end{proof}

\begin{lemma}
Если $A$ -- продуктивна и существует такой вектор $\bar{x}$, что выполняется $\bar{x} \geq A\bar{x}$, то $\bar{x} \geq 0.$
\end{lemma}
\begin{proof}

$$\bar{x} \geq A\bar{x} \geq A^2\bar{x} \geq \dots \geq A^{\nu} \bar{x}$$
$$\bar{x} \geq A^{\nu} \bar{x} \to 0 \text{, при } \nu \to \infty$$
$$\bar{x} \geq 0$$
\end{proof}

\begin{lemma}
Если $A$ -- продуктивна, то $det(E-A) \neq 0.$
\end{lemma}
\begin{proof}
\textit{От противного.}\\
Если $A$ -- продуктивна, но $det(E-A) = 0.$\\
Пусть существует такой вектор $\hat{x} \neq 0$, и пусть $(E-A)\hat{x} = 0 \quad \overset{\mathrm{Lemma 1.2}}{\Longrightarrow} \quad \hat{x} \geq 0$.\\
Теперь возьмем вектор $(-\hat{x}),$ $(E-A)(-\hat{x}) = 0 \quad \overset{\mathrm{Lemma 1.2}}{\Longrightarrow} \quad (-\hat{x}) \geq 0.$
Пришли к противоречию.
\end{proof}

\textit{Необходимость.}

$$(E-A)x=y \quad \forall y \geq 0$$
По Лемме 1.3 $det(E-A) \neq 0$, следовательно решение единственно.
$$(E-A)x \geq 0$$
В силу Леммы 1.2 $x \geq 0.$
\end{proof}

\begin{theorem}

\end{theorem}
\chapter{Линейное программирование}



\chapter{Нелинейное программирование))))))}
\end{document}
