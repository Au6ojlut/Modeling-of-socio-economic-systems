\documentclass[12pt,a4paper,titlepage,oneside]{book}
\usepackage[utf8]{inputenc}
\usepackage[russian]{babel}
\usepackage[OT1]{fontenc}
\usepackage{amsmath}
\usepackage{amsthm}
\usepackage{mathrsfs}
\usepackage{indentfirst}
\usepackage{amsfonts}
\usepackage{bbold}
\usepackage[hidelinks]{hyperref}
\usepackage{amssymb}
\usepackage[left=2cm,right=2cm,top=2cm,bottom=2cm]{geometry}
\title{Функциональный анализ}

%%% COMMANDS
\newcommand{\overbar}[1]{\mkern 1.5mu\overline{\mkern-1.5mu#1\mkern-1.5mu}\mkern 1.5mu}
\newcommand{\argmax}{\operatornamewithlimits{argmax}}
\newcommand{\argmin}{\operatornamewithlimits{argmin}}

%%% THEOREMS
\newtheoremstyle{break}{3pt}{3pt}{\itshape}{}{\bfseries}{.}{\newline}{}

\theoremstyle{definition}
\newtheorem{definition}{Определение}[chapter]

\theoremstyle{plain}
\newtheorem{theorem}{Теорема}[chapter]

\theoremstyle{remark}
\newtheorem{remark}{Замечание}[chapter]

\theoremstyle{remark}
\newtheorem{example}{Пример}[chapter]

\theoremstyle{plain}
\newtheorem{lemma}{Лемма}[chapter]

\theoremstyle{plain}
\newtheorem{corollary}{Следствие}[chapter]

%%% MISC
\setcounter{tocdepth}{1}
\def\labelitemi{--}
\renewcommand{\qedsymbol}{\rule{0.7em}{0.7em}}
\sloppy
\binoppenalty=\maxdimen
\relpenalty=\maxdimen

%%% DOCUMENT
\begin{document}

%%% TITLE PAGE
\begin{titlepage}
\begin{center}

\vfill

Санкт-Петербургский государственный университет\\
\ \\

\vfill

{\large\bf Моделирование социально-экономических систем\\}
\ \\
Лекции 
\vfill

\hfill\vbox
{
\hbox{Доцент кафедры математического моделирования}
\hbox{энергетических систем, кандидат физ.-мат. наук}
\hbox{Александр Юрьевич Крылатов}
}

\vfill

Санкт-Петербург, 2016
\end{center}
\end{titlepage}


\tableofcontents

\chapter{Балансовая модель производства}

\section{Модель <<затрата - выпуск>> (англ. input - output)}

Предположим следующее:
\begin{enumerate}

\item[1)]Количество продукции характеризуется одним числом (у каждого экономического объекта).
\item[2)]Комплектность потребления: для выпуска продукции экономический объект должен получить продукты от других объектов.
\item[3)]Линейность : для увеличения количества производства в $n$ раз, необходимо увеличить ресурс в $n$ раз.
\item[4)]Делимость на конечный продукт и на продукт, который будет использоваться в производстве.

\end{enumerate}

$\\ $

Пусть $n$ --- количество субъектов (экономических субъектов),

$x_i$ --- количество производства продукта $i$,

$\\ $

$x$ = $\left[\begin{array}{crl}
x_1\\ x_2\\ ... \\ x_n
\end{array}\right]$,

$\\ $

$x_{ji}$ --- количество продукта j, необходимого для производства i.

$\\ $

$\begin{cases}
x_{1i} = \alpha_{1i} x_i \\
x_{2i} = \alpha_{2i} x_i \\
... \\
x_{ni} = \alpha_{ni} x_i
\end{cases}$

$\\ $

$A=\left[\begin{array}{crl}
\alpha_{11} & ... & \alpha_{1n} \\
... & ... & ...\\
\alpha_{n1} & ... & \alpha_{nn}
\end{array}\right]$

$\\ $

\begin{definition}
$A$ --- матрица коэффициентов прямых затрат (матрица технологических коэффициентов).

\end{definition}
Матрица $A$ --- положительно полуопределённая ($y^T A y   \geq 0$ для любых ненулевых векторов $y$).


\chapter{Линейное программирование}



\chapter{Нелинейное программирование))))))}
\end{document}
